\newpage
\chapter{Introduction}

With the development of Wi-Fi and WLAN technology, it was accepted widely and used in various industries. Similarly, in location-based services market, the demand for location-based services grows rapidly. If customers have the ability to know their location, there is a competition of features in the customers service. For example, in large shopping centers, location-based service collects the customer spending habits and immediately sends the discount information with the nearest Stores and shopping cart positioning. In the health care industry, tracking is used additional on patient care, monitoring and control for expensive medical equipment. Further applications are indoor environments with a lot of people, such as airport, library and concert halls. In Emergency situation it supports escape navigation and real-time location of inquiries.

The traditional GPS satellite positioning can provide precisely positioning only to relatively open, none high-rise buildings, can’t be used indoors and it has higher power consumption.Therefore, in intensive indoor positioning goals, positioning system based on Wi-Fi technology becomes the best choice. Through the Wi-Fi network location, there is a much better control of systems—based services industries and range, and improve positioning accuracy, reduce deployment costs, improve equipment utilization, enhance rescue capabilities to respond to emergencies, it has important social significance.

This thesis describes about the Wi-Fi and WLAN technology and learn the basic principles, the underlying communication protocol and web-based wireless positioning technology, focusing on analysis of the positioning technology based on SNMP. It´s developed as a \textbf{guidance system} for mobile devices to evacuate people from buildings. The application runs on mobile terminal without any additional software or APP, requires only the web browser. In case of an emergency, the web page through HTTP Redirect technology opens a navigation page.

To achieve this goal, a python project named \textbf{LoYiW} has been implemented. It contains a graphical user interface for management, a simple HTTP server and a bash script for server administration. The final chapter briefly summarize the obtained results and pointed out that the function not yet been achieved,and make a future based on the research results of this subject.